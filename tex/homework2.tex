\documentclass{ctexart}
\usepackage{amsmath}
\usepackage{geometry}
\geometry{a4paper, margin=2.5cm}

\title{年金}
\date{}

\begin{document}
\maketitle

% 1
\textbf{1}
如果货币价值 $7.5\%$,计算一项一年 $4200$ 美元 $6$ 年期限的普通年金的累积价值。

 $$
FV = P\cdot\frac{(1+r)^{n}-1}{r}=30424.8
$$ 

% 3
\textbf{3}
如果Gabe每个季度末给他的存款基金存人450美元期限6年,要是货币价值6.75\%,
第六年末他将能储蓄多少钱?

 $$
FV = P\cdot\frac{(1+r)^{n}-1}{r}=12829.5
$$ 

% 4
\textbf{4}
如果Brenda在每个月末捐赠630美元到她的退休账户,该账户支付8.75\%半年复合利
率,当她从捐赠开始20年后退休时,她将有多少钱?

 $$
(1+r_{m})^{6}=1+0.04375\ \Rightarrow\ r_{m}=(1.04375)^{1/6}-1=0.007166
$$ 

 $$
FV=P\cdot\frac{(1+r_{m})^{240}-1}{r_{m}}=371196
$$ 

% 7
\textbf{7}
一项年金每半年末期付款7500美元在一个11.5\%按年复合的有息账户10年,年金现值是多少?

 $$
PV=P\cdot\frac{1-(1+r)^{-n}}{r}=86430
$$ 

% 8
\textbf{8}
一对夫妇想在三年内翻修他们的房子。他们需要27000美元,他们计划用按月支付的方式存款在一个账户,该账户按月复合利率为8.5\%,他们的月存款数将是多少?
$$
FV = P \cdot \frac{(1+r)^n - 1}{r}
\Rightarrow
P = \frac{FV \cdot r}{(1+r)^n - 1}
$$

$$
P = \frac{27000 \cdot 0.007083}{1.007083^{36} - 1}
= \frac{191.24}{0.2803}
= 682.5
$$ 

% 9
\textbf{9}
Wayne想为他的女儿建一个每月支付,他女儿在将来四年计划生活在另一个州。他为她
的自助银行支付存入40000美元。如果银行支付7.25\%按半年复合利率,她每月将收
到多少钱?

$$
PV = P \cdot \frac{1 - (1+r_m)^{-n}}{r_m} \cdot (1+r_m)
\quad (n=48)
$$

$$
P = \frac{40000}{ \frac{1 - 1.005953^{-48}}{0.005953} \cdot 1.005953 }
= \frac{40000}{42.42}
= 942.9
$$ 

% 10
\textbf{10}
Rosemary向往买一幢价值300O00美元的房屋远跳阿尔卑斯山脉。她能在一个支付
15\%按月复合利率的账户每月存款10000美元,等待将花去多久时间?

$$
FV = P \cdot \frac{(1+r)^n - 1}{r}
\Rightarrow
\frac{FV \cdot r}{P} + 1 = (1+r)^n
$$

$$
n = \frac{\ln 1.375}{\ln 1.0125}
= 25.6 
$$

\textbf{13}
如果年金的将来值是35507.50美元,按季度支付是1750美元期限9年,年金利率是多少?

$$
FV = P \cdot \frac{(1+r)^{n} - 1}{r}
\Rightarrow
\frac{FV}{P} = \frac{(1+r)^{36} - 1}{r} = 20.29
$$

$$
r \approx 0.025 \; \text{每季度}
\Rightarrow
\text{年利率} \approx 4 \times 0.025 = 10.0\%
$$

% 17
\textbf{17}
75000美元抵押被以9\%得到。它应该20年内被支付。求每个月初的支付额。

$$
PV = P \cdot \frac{1 - (1+r)^{-n}}{r} \cdot (1+r)
$$

$$
P = \frac{75000}{ \frac{1 - 1.0075^{-240}}{0.0075} \cdot 1.0075 }
= \frac{75000}{111.42}
= 673.1
$$

% 19
\textbf{19}
Jen在不久的将来需要20000美元,她在每个月初向一个按年复合付息15\%的账户存款875美元,何时才能实现20000美元?

$$
FV = P \cdot \frac{(1+r)^n - 1}{r} \cdot (1+r)
\Rightarrow
\frac{FV \cdot r}{P(1+r)} + 1 = (1+r)^n
$$

$$
n = \frac{\ln 1.282}{\ln 1.0125}
= 19.8 
$$

\end{document}
